\documentclass[12pt,a4paper]{article} 
%% packages for bibliography
\renewcommand*{\nameyeardelim}{\addcomma\space}
%% adding dots in contents
\usepackage{tocloft}
\renewcommand{\cftsecleader}{\cftdotfill{\cftdotsep}}
%% centering sections
\usepackage{sectsty} 
\sectionfont{\centering}
%% adding dot after section number
\usepackage{titlesec}
\titlelabel{\thetitle.\quad}

%% making section headings bold

\usepackage[affil-it]{authblk}
\usepackage[utf8]{inputenc}
\usepackage{fancyhdr} %% package required for setting page numbering
\pagestyle{myheadings}
\fancyhf{}
\fancyhead[R]{\thepage}
\usepackage{indentfirst} %% add indentation at first paragraphs
\renewcommand{\baselinestretch}{1.5} 
\usepackage{geometry}
\geometry{ %% page margins
    a4paper,
    left=30mm,
    right=15mm,
    top=20mm,
    bottom=20mm,
}
\setlength{\parindent}{1cm}
\usepackage[ukrainian]{babel}
\usepackage{graphicx}
\usepackage{listings} %% for adding code snippets
\usepackage{float} % for fixing figures
\usepackage{csvsimple}
\author{}
\date{}

\begin{document}
\section*{Висновки}
\begin{center}
\csvautotabular{Stats/t_tests_pvals.csv}
\end{center}

%Вище можна бачити таблицю зі значеннями статистичної значимості різниці в середніх амплітудах
%для різних груп. Для всіх пар була знайдена статистично значима різниця, крім пари окситоцин і 
%окситоцин + челеретрин.
%
%По отриманих ерорбарах видно, що поодинці челеретрин пригнічує скорочення міометрію (зменшення
%амплітуди, площі), а окситоцин підсилює скорочення міометрію (збільшення амплітуди, площі).
%Пов'язано це з тим, що челеретрин є інгібітором протеїнкінази С, а окситоцин активує протеїнкіназу
%С і збільшує внутрішньоклітинну концентрацію $Ca^{2+}$. Челеретрин здатний проходити через 
%цитоплазматичну мембрану і безпосередньо інгібувати 
%протеїнкіназу С, таке інгібування відбувається досить швидко. Окситоцин діє на протеїнкіназу
%через G-білок: дисоціація $\alpha$-субодиниці, активація фосфоліпази С, розщеплення PIP$_2$ на
%DAG та IP$_3$, IP$_3$ відкриває кальцієві канали на ЕПР, виходить $Ca^{2+}$, кальцій зв'язується
%з кальмодуліном і разом з DAG цей комплекс активує PKC. За раухнок того, що тут в цьому каскаді 
%багато ланок, окситоцин підвищує амплітуду скорочення з певною часовою затримкою.

%PKC підвищує амлітуду скорочення за рахунок того, що вона фосфорилює проміжні філаменти, білок
%кальдесмон, який в дефосфорильованому стані інгібує зв'язування міозину з актином, інші 
%актинзв'язуючі білки. Якраз за рахунок фосфорилювання кальдесмону, скоріш за все, збільшується
%амплітуда скорочень. 
%
%Тоді стає зрозумілим той факт, що челеретрин зменшує амлітуду скорочень, а окситоцин збільшує, тому
%що вони впливають на протеїнкіназу С. Але виникає питання, чому сумарна дія окситоцину і челеретрину
%дає загалом підсилюючий ефект? Тут точно сказати важко, але можливе таке пояснення: якусь частину
%протеїнкінази С окситоцин все таки активує і відбувається "interplay"\ між ефектом інгібування
%челеретрина і активацією окситоцину. Але також відомо, що челеретрин інгібує $Ca^{2+}$-АТФази на
%мембрані ЕПР і плазматичній мембрані. А тому за рахунок того, що окситоцин підвищує, за описаним
%вище механізмом, концентрацію $Ca^{2+}$, а челеретрин зменшує виведення $Ca^{2+}$ з цитозолю, 
%спостерігається підвищення амплітуди скорочення (за рахунок активності кінази легких ланцюгів міозину). Таким чином, ці дві речовини мають взаємодоповнюючу дію.



\end{document}
